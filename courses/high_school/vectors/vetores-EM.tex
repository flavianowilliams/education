% Aula de mec�nica dos fluidos para o curso de F�sica II do ensino m�dio
%
\documentclass[brazil,aspectratio=169]{beamer}
%
%carregando pre�mbulo
% Standard packages

\usepackage[latin1]{inputenc}
\usepackage{times}
\usepackage[LGR,T1]{fontenc}
\usepackage[portuguese]{babel}
\usepackage{subcaption}
\usepackage{enumitem}
\usepackage{graphicx}
\usepackage[justifying]{ragged2e}
\usepackage{multicol}
\usepackage{color}
\usepackage{mathtools}
\usepackage{hyperref}
\usepackage{cancel}
\usepackage{pifont}
\usepackage[output-decimal-marker={,}]{siunitx}
%\usepackage{animate}
\usepackage{tikz}
%\usepackage{geometry}
\usepackage{pgfplots,tikz-3dplot,circuitikz,bm,tkz-euclide,tkz-linknodes}
\usepackage{tkz-fct,tkz-base}

% Setup appearance:

\usetheme{Ilmenau}
%\usefonttheme{serif}
\usefonttheme[onlylarge]{structurebold}
\setbeamerfont*{frametitle}{size=\normalsize,series=\bfseries}
\setbeamertemplate{navigation symbols}{}
\setbeamercovered{transparent}
\setlength{\columnseprule}{1pt} 
\def\columnseprulecolor{\color{blue}} 

%%%%%%%%%%%%%%%%%%%%%%%%%%%%%% LyX specific LaTeX commands.
\DeclareRobustCommand{\greektext}{%
\fontencoding{LGR}\selectfont\def\encodingdefault{LGR}}
\DeclareRobustCommand{\textgreek}[1]{\leavevmode{\greektext #1}}
\DeclareFontEncoding{LGR}{}{}
\DeclareTextSymbol{\~}{LGR}{126}

% Setup new commands:

\renewcommand{\raggedright}{\leftskip=0pt \rightskip=0pt plus 0cm} 

\newcommand\Warning{%
 \makebox[1.4em][c]{%
 \makebox[0pt][c]{\raisebox{.1em}{\small\textrm{!}}}%
 \makebox[0pt][c]{\color{red}\Large$\bigtriangleup$}}}%

\newcommand\doubt{%
 \makebox[1.4em][c]{%
 \makebox[0pt][c]{\small?}%
 \makebox[0pt][c]{\raisebox{.1em}{\color{red}\Large$\bigtriangledown$}}}}%

\newcommand{\eye}[4]% size, x, y, rotation
{   \draw[rotate around={#4:(#2,#3)}] (#2,#3) -- ++(-.5*55:#1) (#2,#3) -- ++(.5*55:#1);
	\draw (#2,#3) ++(#4+55:.75*#1) arc (#4+55:#4-55:.75*#1);
	% IRIS
	\draw[fill=gray] (#2,#3) ++(#4+55/3:.75*#1) arc (#4+180-55:#4+180+55:.28*#1);
	%PUPIL, a filled arc 
	\draw[fill=black] (#2,#3) ++(#4+55/3:.75*#1) arc (#4+55/3:#4-55/3:.75*#1);
}

% Setup TikZ

\usepgfplotslibrary{fillbetween}
\usetikzlibrary{arrows,3d,calc,shadows,patterns,math,fit,shapes,optics,hobby}
\usetikzlibrary{automata,positioning,math,fit}
\usepgfplotslibrary{fillbetween}

\tikzset{>=stealth, thick, global scale/.style={scale=#1,every node/.style={scale=#1}}}
\tikzset{mid arrow/.style={postaction={decorate,decoration={markings,mark=at position .5 with {\arrow[#1]{stealth}}}}}}

%\usetkzobj{all}

\pgfplotsset{compat=1.15}

% Setup paths

\graphicspath{{imagens/}}

%Autor

\author[Prof. Flaviano W. Fernandes]
{\textcolor{green!50!black}{Flaviano Williams Fernandes}}

\institute[IFPR-Irati]
{Instituto Federal do Paran� \\ Campus Irati}

\justifying


%T�tulo

\title[Mec�nica-VET]{Vetores}

% The main document

\begin{document}

\begin{frame}
  \titlepage
\end{frame}

\begin{frame}{Sum�rio}
  \tableofcontents
\end{frame}

\section{Grandezas escalares e vetoriais}

\subsection{}

\begin{frame}{Grandezas escalares}
\end{frame}

\begin{frame}{Espa�o percorrido}
\end{frame}

\begin{frame}{Grandezas vetoriais}
\end{frame}

\begin{frame}{Deslocamento}
\end{frame}

\section{Soma de vetores}

\subsection{}

\begin{frame}{Resultante de dois vetores}
\end{frame}

\begin{frame}{Resultante entre dois vetores e a regra do paralelogramo}
\end{frame}

\begin{frame}{Resultante de v�rios vetores e a regra do pol�gono fechado}
\end{frame}

\begin{frame}{Componentes de um vetor}
\end{frame}

\section{Ap�ndice}

\subsection{}

\begin{frame}{Alfabeto grego}
	\begin{columns}
		\begin{column}[c]{0.5\textwidth}
			\begin{table}
				\begin{tabular}[c]{ccc}
					Alfa & $A$ & $\alpha$\\
					Beta & $B$ & $\beta$\\
					Gama & $\Gamma$ & $\gamma$\\
					Delta & $\Delta$ & $\delta$\\
					Eps�lon & $E$ & $\epsilon$,$\varepsilon$\\
					Zeta & $Z$ & $\zeta$\\
					Eta & $H$ & $\eta$\\
					Teta & $\Theta$ & $\theta$\\
					Iota & $I$ & $\iota$\\
					Capa & $K$ & $\kappa$\\
					Lambda & $\Lambda$ & $\lambda$\\
					Mi & $M$ & $\mu$
				\end{tabular}
			\end{table}
		\end{column}
		\begin{column}[c]{0.5\textwidth}
	\begin{table}
		\begin{tabular}[c]{ccc}
			Ni & $N$ & $\nu$\\
			Csi & $\Xi$ & $\xi$\\
			�micron & $O$ & $o$\\
			Pi & $\Pi$ & $\pi$\\
			R� & $P$ & $\rho$\\
			Sigma & $\Sigma$ & $\sigma$\\
			Tau & $T$ & $\tau$\\
			�psilon & $\Upsilon$ & $\upsilon$\\
			Fi & $\Phi$ & $\phi$,$\varphi$\\
			Qui & $X$ & $\chi$\\
			Psi & $\Psi$ & $\psi$\\
			�mega & $\Omega$ & $\omega$
		\end{tabular}
	\end{table}
\end{column}
	\end{columns}
\end{frame}

\begin{frame}{Refer�ncias e observa��es\footnote{Este material est� sujeito a modifica��es. Recomenda-se acompanhamento permanente.}}
	\bibliographystyle{jurabib}
	\begin{thebibliography}{9}
		\bibitem{alvarenga}A. M�ximo, B. Alvarenga, C. Guimar�es, F�sica. Contexto e aplica��es, v.1, 2.ed., S�o Paulo, Scipione (2016)
		\bibitem{sistemametrico} \url{https://sistemametricodecimal.wordpress.com/2016/07/12/objetivos/}
	\end{thebibliography}

\vspace*{1cm}
\begin{center}
	Esta apresenta��o est� dispon�vel para download no endere�o\\
	\href{https://flavianowilliams.github.io/education}{\textcolor{blue}{https://flavianowilliams.github.io/education}}
\end{center}

\end{frame}

\end{document}
